% Options for packages loaded elsewhere
\PassOptionsToPackage{unicode}{hyperref}
\PassOptionsToPackage{hyphens}{url}
%
\documentclass[
]{article}
\usepackage{amsmath,amssymb}
\usepackage{lmodern}
\usepackage{iftex}
\ifPDFTeX
  \usepackage[T1]{fontenc}
  \usepackage[utf8]{inputenc}
  \usepackage{textcomp} % provide euro and other symbols
\else % if luatex or xetex
  \usepackage{unicode-math}
  \defaultfontfeatures{Scale=MatchLowercase}
  \defaultfontfeatures[\rmfamily]{Ligatures=TeX,Scale=1}
\fi
% Use upquote if available, for straight quotes in verbatim environments
\IfFileExists{upquote.sty}{\usepackage{upquote}}{}
\IfFileExists{microtype.sty}{% use microtype if available
  \usepackage[]{microtype}
  \UseMicrotypeSet[protrusion]{basicmath} % disable protrusion for tt fonts
}{}
\makeatletter
\@ifundefined{KOMAClassName}{% if non-KOMA class
  \IfFileExists{parskip.sty}{%
    \usepackage{parskip}
  }{% else
    \setlength{\parindent}{0pt}
    \setlength{\parskip}{6pt plus 2pt minus 1pt}}
}{% if KOMA class
  \KOMAoptions{parskip=half}}
\makeatother
\usepackage{xcolor}
\IfFileExists{xurl.sty}{\usepackage{xurl}}{} % add URL line breaks if available
\IfFileExists{bookmark.sty}{\usepackage{bookmark}}{\usepackage{hyperref}}
\hypersetup{
  pdftitle={epicov\_pre\_iot},
  pdfauthor={Matheus},
  hidelinks,
  pdfcreator={LaTeX via pandoc}}
\urlstyle{same} % disable monospaced font for URLs
\usepackage[margin=1in]{geometry}
\usepackage{color}
\usepackage{fancyvrb}
\newcommand{\VerbBar}{|}
\newcommand{\VERB}{\Verb[commandchars=\\\{\}]}
\DefineVerbatimEnvironment{Highlighting}{Verbatim}{commandchars=\\\{\}}
% Add ',fontsize=\small' for more characters per line
\usepackage{framed}
\definecolor{shadecolor}{RGB}{248,248,248}
\newenvironment{Shaded}{\begin{snugshade}}{\end{snugshade}}
\newcommand{\AlertTok}[1]{\textcolor[rgb]{0.94,0.16,0.16}{#1}}
\newcommand{\AnnotationTok}[1]{\textcolor[rgb]{0.56,0.35,0.01}{\textbf{\textit{#1}}}}
\newcommand{\AttributeTok}[1]{\textcolor[rgb]{0.77,0.63,0.00}{#1}}
\newcommand{\BaseNTok}[1]{\textcolor[rgb]{0.00,0.00,0.81}{#1}}
\newcommand{\BuiltInTok}[1]{#1}
\newcommand{\CharTok}[1]{\textcolor[rgb]{0.31,0.60,0.02}{#1}}
\newcommand{\CommentTok}[1]{\textcolor[rgb]{0.56,0.35,0.01}{\textit{#1}}}
\newcommand{\CommentVarTok}[1]{\textcolor[rgb]{0.56,0.35,0.01}{\textbf{\textit{#1}}}}
\newcommand{\ConstantTok}[1]{\textcolor[rgb]{0.00,0.00,0.00}{#1}}
\newcommand{\ControlFlowTok}[1]{\textcolor[rgb]{0.13,0.29,0.53}{\textbf{#1}}}
\newcommand{\DataTypeTok}[1]{\textcolor[rgb]{0.13,0.29,0.53}{#1}}
\newcommand{\DecValTok}[1]{\textcolor[rgb]{0.00,0.00,0.81}{#1}}
\newcommand{\DocumentationTok}[1]{\textcolor[rgb]{0.56,0.35,0.01}{\textbf{\textit{#1}}}}
\newcommand{\ErrorTok}[1]{\textcolor[rgb]{0.64,0.00,0.00}{\textbf{#1}}}
\newcommand{\ExtensionTok}[1]{#1}
\newcommand{\FloatTok}[1]{\textcolor[rgb]{0.00,0.00,0.81}{#1}}
\newcommand{\FunctionTok}[1]{\textcolor[rgb]{0.00,0.00,0.00}{#1}}
\newcommand{\ImportTok}[1]{#1}
\newcommand{\InformationTok}[1]{\textcolor[rgb]{0.56,0.35,0.01}{\textbf{\textit{#1}}}}
\newcommand{\KeywordTok}[1]{\textcolor[rgb]{0.13,0.29,0.53}{\textbf{#1}}}
\newcommand{\NormalTok}[1]{#1}
\newcommand{\OperatorTok}[1]{\textcolor[rgb]{0.81,0.36,0.00}{\textbf{#1}}}
\newcommand{\OtherTok}[1]{\textcolor[rgb]{0.56,0.35,0.01}{#1}}
\newcommand{\PreprocessorTok}[1]{\textcolor[rgb]{0.56,0.35,0.01}{\textit{#1}}}
\newcommand{\RegionMarkerTok}[1]{#1}
\newcommand{\SpecialCharTok}[1]{\textcolor[rgb]{0.00,0.00,0.00}{#1}}
\newcommand{\SpecialStringTok}[1]{\textcolor[rgb]{0.31,0.60,0.02}{#1}}
\newcommand{\StringTok}[1]{\textcolor[rgb]{0.31,0.60,0.02}{#1}}
\newcommand{\VariableTok}[1]{\textcolor[rgb]{0.00,0.00,0.00}{#1}}
\newcommand{\VerbatimStringTok}[1]{\textcolor[rgb]{0.31,0.60,0.02}{#1}}
\newcommand{\WarningTok}[1]{\textcolor[rgb]{0.56,0.35,0.01}{\textbf{\textit{#1}}}}
\usepackage{graphicx}
\makeatletter
\def\maxwidth{\ifdim\Gin@nat@width>\linewidth\linewidth\else\Gin@nat@width\fi}
\def\maxheight{\ifdim\Gin@nat@height>\textheight\textheight\else\Gin@nat@height\fi}
\makeatother
% Scale images if necessary, so that they will not overflow the page
% margins by default, and it is still possible to overwrite the defaults
% using explicit options in \includegraphics[width, height, ...]{}
\setkeys{Gin}{width=\maxwidth,height=\maxheight,keepaspectratio}
% Set default figure placement to htbp
\makeatletter
\def\fps@figure{htbp}
\makeatother
\setlength{\emergencystretch}{3em} % prevent overfull lines
\providecommand{\tightlist}{%
  \setlength{\itemsep}{0pt}\setlength{\parskip}{0pt}}
\setcounter{secnumdepth}{-\maxdimen} % remove section numbering
\ifLuaTeX
  \usepackage{selnolig}  % disable illegal ligatures
\fi

\title{epicov\_pre\_iot}
\author{Matheus}
\date{23/08/2021}

\begin{document}
\maketitle

\hypertarget{projeto-epicov---cnafvni}{%
\section{\texorpdfstring{\textbf{Projeto EPICOV -
CNAF/VNI}}{Projeto EPICOV - CNAF/VNI}}\label{projeto-epicov---cnafvni}}

\hypertarget{anuxe1lise-descritivaexploratuxf3ria}{%
\subsection{\texorpdfstring{\textbf{Análise
descritiva/exploratória}}{Análise descritiva/exploratória}}\label{anuxe1lise-descritivaexploratuxf3ria}}

\hypertarget{comparauxe7uxe3o-idade}{%
\subsubsection{\texorpdfstring{\textbf{Comparação
idade}}{Comparação idade}}\label{comparauxe7uxe3o-idade}}

\hypertarget{idade-grupo-cnafvni}{%
\paragraph{\texorpdfstring{\textbf{Idade grupo
CNAF/VNI}}{Idade grupo CNAF/VNI}}\label{idade-grupo-cnafvni}}

\begin{Shaded}
\begin{Highlighting}[]
\FunctionTok{as.data.frame}\NormalTok{(}\FunctionTok{skim}\NormalTok{(df\_pre}\SpecialCharTok{$}\NormalTok{Idade))}
\end{Highlighting}
\end{Shaded}

\begin{verbatim}
##   skim_type skim_variable n_missing complete_rate numeric.mean numeric.sd
## 1   numeric          data         0             1         60.2         15
##   numeric.p0 numeric.p25 numeric.p50 numeric.p75 numeric.p100
## 1         19        51.5          62          70           96
##                               numeric.hist
## 1 <U+2581><U+2583><U+2587><U+2586><U+2582>
\end{verbatim}

\hypertarget{idade-grupo-controle}{%
\paragraph{\texorpdfstring{\textbf{Idade grupo
controle}}{Idade grupo controle}}\label{idade-grupo-controle}}

\begin{Shaded}
\begin{Highlighting}[]
\FunctionTok{as.data.frame}\NormalTok{(}\FunctionTok{skim}\NormalTok{(df\_npre}\SpecialCharTok{$}\NormalTok{Idade))}
\end{Highlighting}
\end{Shaded}

\begin{verbatim}
##   skim_type skim_variable n_missing complete_rate numeric.mean numeric.sd
## 1   numeric          data         0             1         59.5       15.7
##   numeric.p0 numeric.p25 numeric.p50 numeric.p75 numeric.p100
## 1         15          49          61          71           97
##                               numeric.hist
## 1 <U+2581><U+2585><U+2587><U+2587><U+2582>
\end{verbatim}

\hypertarget{teste-de-normalidade}{%
\paragraph{\texorpdfstring{\textbf{Teste de
normalidade}}{Teste de normalidade}}\label{teste-de-normalidade}}

\begin{Shaded}
\begin{Highlighting}[]
\CommentTok{\#Grupo Intervenção}
\FunctionTok{shapiro.test}\NormalTok{(df\_pre}\SpecialCharTok{$}\NormalTok{Idade)}
\end{Highlighting}
\end{Shaded}

\begin{verbatim}
## 
##  Shapiro-Wilk normality test
## 
## data:  df_pre$Idade
## W = 1, p-value = 0.008
\end{verbatim}

\begin{Shaded}
\begin{Highlighting}[]
\FunctionTok{shapiro.test}\NormalTok{(df\_npre}\SpecialCharTok{$}\NormalTok{Idade)}
\end{Highlighting}
\end{Shaded}

\begin{verbatim}
## 
##  Shapiro-Wilk normality test
## 
## data:  df_npre$Idade
## W = 1, p-value = 6e-08
\end{verbatim}

\hypertarget{concluuxedmos-que-ambas-as-amostras-possuem-distribuiuxe7uxe3o-nuxe3o-paramuxe9trica.}{%
\paragraph{\texorpdfstring{\textbf{Concluímos que ambas as amostras
possuem distribuição
não-paramétrica.}}{Concluímos que ambas as amostras possuem distribuição não-paramétrica.}}\label{concluuxedmos-que-ambas-as-amostras-possuem-distribuiuxe7uxe3o-nuxe3o-paramuxe9trica.}}

\hypertarget{comparauxe7uxe3o-idade-1}{%
\paragraph{\texorpdfstring{\textbf{Comparação
Idade}}{Comparação Idade}}\label{comparauxe7uxe3o-idade-1}}

\begin{Shaded}
\begin{Highlighting}[]
\FunctionTok{ggplot}\NormalTok{(df\_pret, }\FunctionTok{aes}\NormalTok{(Idade, }\AttributeTok{fill=}\NormalTok{Grupo)) }\SpecialCharTok{+}
  \FunctionTok{geom\_density}\NormalTok{(}\AttributeTok{alpha=}\FloatTok{0.2}\NormalTok{)}
\end{Highlighting}
\end{Shaded}

\includegraphics{relatorio_epicov_pre_iot_files/figure-latex/unnamed-chunk-3-1.pdf}

\begin{Shaded}
\begin{Highlighting}[]
\FunctionTok{wilcox.test}\NormalTok{(df\_pre}\SpecialCharTok{$}\NormalTok{Idade, df\_npre}\SpecialCharTok{$}\NormalTok{Idade)}
\end{Highlighting}
\end{Shaded}

\begin{verbatim}
## 
##  Wilcoxon rank sum test with continuity correction
## 
## data:  df_pre$Idade and df_npre$Idade
## W = 1e+05, p-value = 0.7
## alternative hypothesis: true location shift is not equal to 0
\end{verbatim}

\hypertarget{conclusuxe3o-nuxe3o-huxe1-diferenuxe7a-entre-ambos-os-grupos.}{%
\paragraph{Conclusão: Não há diferença entre ambos os
grupos.}\label{conclusuxe3o-nuxe3o-huxe1-diferenuxe7a-entre-ambos-os-grupos.}}

\hypertarget{comparauxe7uxe3o---sexo}{%
\subsubsection{\texorpdfstring{\textbf{Comparação -
Sexo}}{Comparação - Sexo}}\label{comparauxe7uxe3o---sexo}}

\hypertarget{sexo---grupo-intervenuxe7uxe3o}{%
\paragraph{\texorpdfstring{\textbf{Sexo - grupo
intervenção}}{Sexo - grupo intervenção}}\label{sexo---grupo-intervenuxe7uxe3o}}

\begin{Shaded}
\begin{Highlighting}[]
\FunctionTok{as.data.frame}\NormalTok{(}\FunctionTok{skim}\NormalTok{(df\_pre}\SpecialCharTok{$}\NormalTok{Sexo))}
\end{Highlighting}
\end{Shaded}

\begin{verbatim}
##   skim_type skim_variable n_missing complete_rate character.min character.max
## 1 character          data         0             1             8             9
##   character.empty character.n_unique character.whitespace
## 1               0                  2                    0
\end{verbatim}

\begin{Shaded}
\begin{Highlighting}[]
\FunctionTok{table}\NormalTok{(df\_pre}\SpecialCharTok{$}\NormalTok{Sexo)}
\end{Highlighting}
\end{Shaded}

\begin{verbatim}
## 
##  feminino masculino 
##       123       148
\end{verbatim}

\begin{Shaded}
\begin{Highlighting}[]
\FunctionTok{prop.table}\NormalTok{(}\FunctionTok{table}\NormalTok{(df\_pre}\SpecialCharTok{$}\NormalTok{Sexo))}
\end{Highlighting}
\end{Shaded}

\begin{verbatim}
## 
##  feminino masculino 
##     0.454     0.546
\end{verbatim}

\begin{Shaded}
\begin{Highlighting}[]
\FunctionTok{ggplot}\NormalTok{(}\AttributeTok{data=}\NormalTok{df\_pre, }\FunctionTok{aes}\NormalTok{(}\AttributeTok{x=}\NormalTok{Sexo, }\AttributeTok{y=}\NormalTok{(..count..)}\SpecialCharTok{*}\DecValTok{100}\SpecialCharTok{/}\FunctionTok{sum}\NormalTok{(..count..), }\AttributeTok{fill=}\NormalTok{Sexo))}\SpecialCharTok{+}
  \FunctionTok{geom\_bar}\NormalTok{()}
\end{Highlighting}
\end{Shaded}

\includegraphics{relatorio_epicov_pre_iot_files/figure-latex/unnamed-chunk-4-1.pdf}

\hypertarget{sexo--grupo-controle}{%
\paragraph{\texorpdfstring{\textbf{Sexo- grupo
controle}}{Sexo- grupo controle}}\label{sexo--grupo-controle}}

\begin{Shaded}
\begin{Highlighting}[]
\FunctionTok{as.data.frame}\NormalTok{(}\FunctionTok{skim}\NormalTok{(df\_npre}\SpecialCharTok{$}\NormalTok{Sexo))}
\end{Highlighting}
\end{Shaded}

\begin{verbatim}
##   skim_type skim_variable n_missing complete_rate character.min character.max
## 1 character          data         0             1             8             9
##   character.empty character.n_unique character.whitespace
## 1               0                  2                    0
\end{verbatim}

\begin{Shaded}
\begin{Highlighting}[]
\FunctionTok{table}\NormalTok{(df\_npre}\SpecialCharTok{$}\NormalTok{Sexo)}
\end{Highlighting}
\end{Shaded}

\begin{verbatim}
## 
##  feminino masculino 
##       308       492
\end{verbatim}

\begin{Shaded}
\begin{Highlighting}[]
\FunctionTok{prop.table}\NormalTok{(}\FunctionTok{table}\NormalTok{(df\_npre}\SpecialCharTok{$}\NormalTok{Sexo))}
\end{Highlighting}
\end{Shaded}

\begin{verbatim}
## 
##  feminino masculino 
##     0.385     0.615
\end{verbatim}

\begin{Shaded}
\begin{Highlighting}[]
\FunctionTok{ggplot}\NormalTok{(}\AttributeTok{data=}\NormalTok{df\_npre, }\FunctionTok{aes}\NormalTok{(}\AttributeTok{x=}\NormalTok{Sexo, }\AttributeTok{y=}\NormalTok{(..count..)}\SpecialCharTok{*}\DecValTok{100}\SpecialCharTok{/}\FunctionTok{sum}\NormalTok{(..count..), }\AttributeTok{fill=}\NormalTok{Sexo))}\SpecialCharTok{+}
  \FunctionTok{geom\_bar}\NormalTok{()}
\end{Highlighting}
\end{Shaded}

\includegraphics{relatorio_epicov_pre_iot_files/figure-latex/unnamed-chunk-5-1.pdf}
\#\#\#\# \textbf{Comparação - Sexo}

\begin{Shaded}
\begin{Highlighting}[]
\NormalTok{t1 }\OtherTok{\textless{}{-}} \FunctionTok{select}\NormalTok{(df\_pre, Sexo, Grupo)}
\NormalTok{t1}
\end{Highlighting}
\end{Shaded}

\begin{verbatim}
## # A tibble: 271 x 2
##    Sexo      Grupo
##    <chr>     <chr>
##  1 masculino Int  
##  2 masculino Int  
##  3 masculino Int  
##  4 masculino Int  
##  5 feminino  Int  
##  6 feminino  Int  
##  7 masculino Int  
##  8 masculino Int  
##  9 masculino Int  
## 10 masculino Int  
## # ... with 261 more rows
\end{verbatim}

\begin{Shaded}
\begin{Highlighting}[]
\NormalTok{t2 }\OtherTok{\textless{}{-}} \FunctionTok{select}\NormalTok{(df\_npre, Sexo, Grupo)}
\NormalTok{t2}
\end{Highlighting}
\end{Shaded}

\begin{verbatim}
## # A tibble: 800 x 2
##    Sexo      Grupo   
##    <chr>     <chr>   
##  1 feminino  Controle
##  2 feminino  Controle
##  3 masculino Controle
##  4 masculino Controle
##  5 masculino Controle
##  6 feminino  Controle
##  7 feminino  Controle
##  8 feminino  Controle
##  9 feminino  Controle
## 10 masculino Controle
## # ... with 790 more rows
\end{verbatim}

\begin{Shaded}
\begin{Highlighting}[]
\NormalTok{t3 }\OtherTok{\textless{}{-}} \FunctionTok{rbind}\NormalTok{(t1, t2)}
\FunctionTok{chisq.test}\NormalTok{(}\FunctionTok{table}\NormalTok{(t3))}
\end{Highlighting}
\end{Shaded}

\begin{verbatim}
## 
##  Pearson's Chi-squared test with Yates' continuity correction
## 
## data:  table(t3)
## X-squared = 4, df = 1, p-value = 0.05
\end{verbatim}

\hypertarget{nuxe3o-observamos-diferenuxe7a-estatuxedsticamente-significativa-entre-ambos-os-grupos-poruxe9m-o-valor-de-p-quase-atinge-o-limite.}{%
\paragraph{Não observamos diferença estatísticamente significativa entre
ambos os grupos, porém, o valor de ``p'' quase atinge o
limite.}\label{nuxe3o-observamos-diferenuxe7a-estatuxedsticamente-significativa-entre-ambos-os-grupos-poruxe9m-o-valor-de-p-quase-atinge-o-limite.}}

\hypertarget{comparauxe7uxe3o-imc}{%
\subsubsection{Comparação IMC}\label{comparauxe7uxe3o-imc}}

\hypertarget{imc---grupo-cnafvni}{%
\paragraph{\texorpdfstring{\textbf{IMC - Grupo
CnAF/VNI}}{IMC - Grupo CnAF/VNI}}\label{imc---grupo-cnafvni}}

\begin{Shaded}
\begin{Highlighting}[]
\FunctionTok{as.data.frame}\NormalTok{(}\FunctionTok{skim}\NormalTok{(df\_pre}\SpecialCharTok{$}\NormalTok{IMC))}
\end{Highlighting}
\end{Shaded}

\begin{verbatim}
##   skim_type skim_variable n_missing complete_rate numeric.mean numeric.sd
## 1   numeric          data        21         0.923         28.6       7.25
##   numeric.p0 numeric.p25 numeric.p50 numeric.p75 numeric.p100
## 1         18          24          27          31           56
##                               numeric.hist
## 1 <U+2587><U+2587><U+2582><U+2581><U+2581>
\end{verbatim}

\begin{Shaded}
\begin{Highlighting}[]
\FunctionTok{ggplot}\NormalTok{(}\AttributeTok{data=}\NormalTok{df\_pre, }\FunctionTok{aes}\NormalTok{(}\AttributeTok{x=}\NormalTok{IMC))}\SpecialCharTok{+}
  \FunctionTok{geom\_density}\NormalTok{()}
\end{Highlighting}
\end{Shaded}

\begin{verbatim}
## Warning: Removed 21 rows containing non-finite values (stat_density).
\end{verbatim}

\includegraphics{relatorio_epicov_pre_iot_files/figure-latex/unnamed-chunk-7-1.pdf}

\hypertarget{imc---grupo-controle}{%
\paragraph{\texorpdfstring{\textbf{IMC - Grupo
controle}}{IMC - Grupo controle}}\label{imc---grupo-controle}}

\begin{Shaded}
\begin{Highlighting}[]
\FunctionTok{as.data.frame}\NormalTok{(}\FunctionTok{skim}\NormalTok{(df\_npre}\SpecialCharTok{$}\NormalTok{IMC))}
\end{Highlighting}
\end{Shaded}

\begin{verbatim}
##   skim_type skim_variable n_missing complete_rate numeric.mean numeric.sd
## 1   numeric          data        78         0.902         27.3       7.59
##   numeric.p0 numeric.p25 numeric.p50 numeric.p75 numeric.p100
## 1         14          23          25          29           97
##                               numeric.hist
## 1 <U+2587><U+2582><U+2581><U+2581><U+2581>
\end{verbatim}

\begin{Shaded}
\begin{Highlighting}[]
\FunctionTok{ggplot}\NormalTok{(}\AttributeTok{data=}\NormalTok{df\_npre, }\FunctionTok{aes}\NormalTok{(}\AttributeTok{x=}\NormalTok{IMC))}\SpecialCharTok{+}
  \FunctionTok{geom\_density}\NormalTok{()}
\end{Highlighting}
\end{Shaded}

\begin{verbatim}
## Warning: Removed 78 rows containing non-finite values (stat_density).
\end{verbatim}

\includegraphics{relatorio_epicov_pre_iot_files/figure-latex/unnamed-chunk-8-1.pdf}

\hypertarget{comparauxe7uxe3o}{%
\paragraph{\texorpdfstring{\textbf{Comparação}}{Comparação}}\label{comparauxe7uxe3o}}

\begin{Shaded}
\begin{Highlighting}[]
\FunctionTok{ggplot}\NormalTok{(}\AttributeTok{data=}\NormalTok{df\_pret, }\FunctionTok{aes}\NormalTok{(}\AttributeTok{x=}\NormalTok{IMC, }\AttributeTok{fill=}\NormalTok{Grupo)) }\SpecialCharTok{+}
  \FunctionTok{geom\_density}\NormalTok{(}\AttributeTok{alpha=}\FloatTok{0.2}\NormalTok{)}
\end{Highlighting}
\end{Shaded}

\begin{verbatim}
## Warning: Removed 99 rows containing non-finite values (stat_density).
\end{verbatim}

\includegraphics{relatorio_epicov_pre_iot_files/figure-latex/unnamed-chunk-9-1.pdf}

\begin{Shaded}
\begin{Highlighting}[]
\FunctionTok{shapiro.test}\NormalTok{(df\_pre}\SpecialCharTok{$}\NormalTok{IMC)}
\end{Highlighting}
\end{Shaded}

\begin{verbatim}
## 
##  Shapiro-Wilk normality test
## 
## data:  df_pre$IMC
## W = 0.9, p-value = 2e-13
\end{verbatim}

\begin{Shaded}
\begin{Highlighting}[]
\FunctionTok{shapiro.test}\NormalTok{(df\_npre}\SpecialCharTok{$}\NormalTok{IMC)}
\end{Highlighting}
\end{Shaded}

\begin{verbatim}
## 
##  Shapiro-Wilk normality test
## 
## data:  df_npre$IMC
## W = 0.7, p-value <2e-16
\end{verbatim}

\begin{Shaded}
\begin{Highlighting}[]
\FunctionTok{wilcox.test}\NormalTok{(df\_pre}\SpecialCharTok{$}\NormalTok{IMC, df\_npre}\SpecialCharTok{$}\NormalTok{IMC)}
\end{Highlighting}
\end{Shaded}

\begin{verbatim}
## 
##  Wilcoxon rank sum test with continuity correction
## 
## data:  df_pre$IMC and df_npre$IMC
## W = 1e+05, p-value = 0.002
## alternative hypothesis: true location shift is not equal to 0
\end{verbatim}

\hypertarget{ambos-os-grupos-possuem-distribuiuxe7uxe3o-nuxe3o-paramuxe9trica.-conclusuxe3o-diferenuxe7a-estatuxedsticamente-significativa}{%
\paragraph{\texorpdfstring{\textbf{Ambos os grupos possuem distribuição
não-paramétrica. Conclusão: Diferença estatísticamente
significativa?!}}{Ambos os grupos possuem distribuição não-paramétrica. Conclusão: Diferença estatísticamente significativa?!}}\label{ambos-os-grupos-possuem-distribuiuxe7uxe3o-nuxe3o-paramuxe9trica.-conclusuxe3o-diferenuxe7a-estatuxedsticamente-significativa}}

\hypertarget{comparauxe7uxe3o-inuxedcio-dos-sintomas---internauxe7uxe3o-hospiptalar}{%
\subsubsection{\texorpdfstring{\textbf{Comparação início dos sintomas -
internação
hospiptalar}}{Comparação início dos sintomas - internação hospiptalar}}\label{comparauxe7uxe3o-inuxedcio-dos-sintomas---internauxe7uxe3o-hospiptalar}}

\hypertarget{grupo-cnafvni}{%
\paragraph{\texorpdfstring{\textbf{Grupo
CnAF/VNI}}{Grupo CnAF/VNI}}\label{grupo-cnafvni}}

\begin{Shaded}
\begin{Highlighting}[]
\FunctionTok{as.data.frame}\NormalTok{(}\FunctionTok{skim}\NormalTok{(df\_pre}\SpecialCharTok{$}\StringTok{\textasciigrave{}}\AttributeTok{Data do início dos sintomas}\StringTok{\textasciigrave{}}\NormalTok{))}
\end{Highlighting}
\end{Shaded}

\begin{verbatim}
##   skim_type skim_variable n_missing complete_rate POSIXct.min POSIXct.max
## 1   POSIXct          data         0             1  2020-03-10  2020-06-27
##   POSIXct.median POSIXct.n_unique
## 1     2020-05-10               91
\end{verbatim}

\begin{Shaded}
\begin{Highlighting}[]
\FunctionTok{as.data.frame}\NormalTok{(}\FunctionTok{skim}\NormalTok{(df\_pre}\SpecialCharTok{$}\StringTok{\textasciigrave{}}\AttributeTok{Data de admissão hospitalar no HC}\StringTok{\textasciigrave{}}\NormalTok{))}
\end{Highlighting}
\end{Shaded}

\begin{verbatim}
##   skim_type skim_variable n_missing complete_rate POSIXct.min POSIXct.max
## 1   POSIXct          data         0             1  2020-03-17  2020-06-30
##   POSIXct.median POSIXct.n_unique
## 1     2020-05-17               87
\end{verbatim}

\begin{Shaded}
\begin{Highlighting}[]
\FunctionTok{mean}\NormalTok{(}\FunctionTok{difftime}\NormalTok{(df\_pre}\SpecialCharTok{$}\StringTok{\textasciigrave{}}\AttributeTok{Data de admissão hospitalar no HC}\StringTok{\textasciigrave{}}\NormalTok{, df\_pre}\SpecialCharTok{$}\StringTok{\textasciigrave{}}\AttributeTok{Data do início dos sintomas}\StringTok{\textasciigrave{}}\NormalTok{, }\AttributeTok{units =} \StringTok{"days"}\NormalTok{))}
\end{Highlighting}
\end{Shaded}

\begin{verbatim}
## Time difference of 7.35 days
\end{verbatim}

\hypertarget{grupo-controle}{%
\paragraph{\texorpdfstring{\textbf{Grupo
controle}}{Grupo controle}}\label{grupo-controle}}

\begin{Shaded}
\begin{Highlighting}[]
\FunctionTok{as.data.frame}\NormalTok{(}\FunctionTok{skim}\NormalTok{(df\_npre}\SpecialCharTok{$}\StringTok{\textasciigrave{}}\AttributeTok{Data do início dos sintomas}\StringTok{\textasciigrave{}}\NormalTok{))}
\end{Highlighting}
\end{Shaded}

\begin{verbatim}
##   skim_type skim_variable n_missing complete_rate POSIXct.min POSIXct.max
## 1   POSIXct          data         0             1  2020-02-28  2020-06-27
##   POSIXct.median POSIXct.n_unique
## 1     2020-05-01              107
\end{verbatim}

\begin{Shaded}
\begin{Highlighting}[]
\FunctionTok{as.data.frame}\NormalTok{(}\FunctionTok{skim}\NormalTok{(df\_npre}\SpecialCharTok{$}\StringTok{\textasciigrave{}}\AttributeTok{Data de admissão hospitalar no HC}\StringTok{\textasciigrave{}}\NormalTok{))}
\end{Highlighting}
\end{Shaded}

\begin{verbatim}
##   skim_type skim_variable n_missing complete_rate POSIXct.min POSIXct.max
## 1   POSIXct          data         0             1  2020-03-30  2020-06-30
##   POSIXct.median POSIXct.n_unique
## 1     2020-05-10               93
\end{verbatim}

\begin{Shaded}
\begin{Highlighting}[]
\FunctionTok{mean}\NormalTok{(}\FunctionTok{difftime}\NormalTok{(df\_npre}\SpecialCharTok{$}\StringTok{\textasciigrave{}}\AttributeTok{Data de admissão hospitalar no HC}\StringTok{\textasciigrave{}}\NormalTok{, df\_npre}\SpecialCharTok{$}\StringTok{\textasciigrave{}}\AttributeTok{Data do início dos sintomas}\StringTok{\textasciigrave{}}\NormalTok{, }\AttributeTok{units =} \StringTok{"days"}\NormalTok{))}
\end{Highlighting}
\end{Shaded}

\begin{verbatim}
## Time difference of 8.77 days
\end{verbatim}

\hypertarget{inuxedcio-dos-sintomas---internauxe7uxe3o-uti}{%
\subsubsection{\texorpdfstring{\textbf{Início dos sintomas - Internação
UTI}}{Início dos sintomas - Internação UTI}}\label{inuxedcio-dos-sintomas---internauxe7uxe3o-uti}}

\hypertarget{grupo-cnafvni-1}{%
\paragraph{\texorpdfstring{\textbf{Grupo
CnAF/VNI}}{Grupo CnAF/VNI}}\label{grupo-cnafvni-1}}

\begin{Shaded}
\begin{Highlighting}[]
\FunctionTok{as.data.frame}\NormalTok{(}\FunctionTok{skim}\NormalTok{(df\_pre}\SpecialCharTok{$}\StringTok{\textasciigrave{}}\AttributeTok{Data do início dos sintomas}\StringTok{\textasciigrave{}}\NormalTok{))}
\end{Highlighting}
\end{Shaded}

\begin{verbatim}
##   skim_type skim_variable n_missing complete_rate POSIXct.min POSIXct.max
## 1   POSIXct          data         0             1  2020-03-10  2020-06-27
##   POSIXct.median POSIXct.n_unique
## 1     2020-05-10               91
\end{verbatim}

\begin{Shaded}
\begin{Highlighting}[]
\FunctionTok{as.data.frame}\NormalTok{(}\FunctionTok{skim}\NormalTok{(df\_pre}\SpecialCharTok{$}\StringTok{\textasciigrave{}}\AttributeTok{Data de admissão UTI}\StringTok{\textasciigrave{}}\NormalTok{))}
\end{Highlighting}
\end{Shaded}

\begin{verbatim}
##   skim_type skim_variable n_missing complete_rate POSIXct.min POSIXct.max
## 1   POSIXct          data         0             1  2020-03-30  2020-06-30
##   POSIXct.median POSIXct.n_unique
## 1     2020-05-21               86
\end{verbatim}

\begin{Shaded}
\begin{Highlighting}[]
\FunctionTok{mean}\NormalTok{(}\FunctionTok{difftime}\NormalTok{(df\_pre}\SpecialCharTok{$}\StringTok{\textasciigrave{}}\AttributeTok{Data de admissão UTI}\StringTok{\textasciigrave{}}\NormalTok{, df\_pre}\SpecialCharTok{$}\StringTok{\textasciigrave{}}\AttributeTok{Data do início dos sintomas}\StringTok{\textasciigrave{}}\NormalTok{, }\AttributeTok{units =} \StringTok{"days"}\NormalTok{))}
\end{Highlighting}
\end{Shaded}

\begin{verbatim}
## Time difference of 9.82 days
\end{verbatim}

\hypertarget{grupo-controle-1}{%
\paragraph{Grupo controle}\label{grupo-controle-1}}

\begin{Shaded}
\begin{Highlighting}[]
\FunctionTok{as.data.frame}\NormalTok{(}\FunctionTok{skim}\NormalTok{(df\_npre}\SpecialCharTok{$}\StringTok{\textasciigrave{}}\AttributeTok{Data do início dos sintomas}\StringTok{\textasciigrave{}}\NormalTok{))}
\end{Highlighting}
\end{Shaded}

\begin{verbatim}
##   skim_type skim_variable n_missing complete_rate POSIXct.min POSIXct.max
## 1   POSIXct          data         0             1  2020-02-28  2020-06-27
##   POSIXct.median POSIXct.n_unique
## 1     2020-05-01              107
\end{verbatim}

\begin{Shaded}
\begin{Highlighting}[]
\FunctionTok{as.data.frame}\NormalTok{(}\FunctionTok{skim}\NormalTok{(df\_npre}\SpecialCharTok{$}\StringTok{\textasciigrave{}}\AttributeTok{Data de admissão UTI}\StringTok{\textasciigrave{}}\NormalTok{))}
\end{Highlighting}
\end{Shaded}

\begin{verbatim}
##   skim_type skim_variable n_missing complete_rate POSIXct.min POSIXct.max
## 1   POSIXct          data         0             1  2020-03-30  2020-06-30
##   POSIXct.median POSIXct.n_unique
## 1     2020-05-11               93
\end{verbatim}

\begin{Shaded}
\begin{Highlighting}[]
\FunctionTok{mean}\NormalTok{(}\FunctionTok{difftime}\NormalTok{(df\_npre}\SpecialCharTok{$}\StringTok{\textasciigrave{}}\AttributeTok{Data de admissão UTI}\StringTok{\textasciigrave{}}\NormalTok{, df\_npre}\SpecialCharTok{$}\StringTok{\textasciigrave{}}\AttributeTok{Data do início dos sintomas}\StringTok{\textasciigrave{}}\NormalTok{, }\AttributeTok{units =} \StringTok{"days"}\NormalTok{))}
\end{Highlighting}
\end{Shaded}

\begin{verbatim}
## Time difference of 9.76 days
\end{verbatim}

\hypertarget{internauxe7uxe3o-hospitalar---internauxe7uxe3o-uti}{%
\subsubsection{\texorpdfstring{\textbf{Internação hospitalar -
internação
UTI}}{Internação hospitalar - internação UTI}}\label{internauxe7uxe3o-hospitalar---internauxe7uxe3o-uti}}

\hypertarget{grupo-cnafvni-2}{%
\paragraph{\texorpdfstring{\textbf{Grupo
CnAF/VNI}}{Grupo CnAF/VNI}}\label{grupo-cnafvni-2}}

\begin{Shaded}
\begin{Highlighting}[]
\FunctionTok{as.data.frame}\NormalTok{(}\FunctionTok{skim}\NormalTok{(df\_pre}\SpecialCharTok{$}\StringTok{\textasciigrave{}}\AttributeTok{Data de admissão hospitalar no HC}\StringTok{\textasciigrave{}}\NormalTok{))}
\end{Highlighting}
\end{Shaded}

\begin{verbatim}
##   skim_type skim_variable n_missing complete_rate POSIXct.min POSIXct.max
## 1   POSIXct          data         0             1  2020-03-17  2020-06-30
##   POSIXct.median POSIXct.n_unique
## 1     2020-05-17               87
\end{verbatim}

\begin{Shaded}
\begin{Highlighting}[]
\FunctionTok{as.data.frame}\NormalTok{(}\FunctionTok{skim}\NormalTok{(df\_pre}\SpecialCharTok{$}\StringTok{\textasciigrave{}}\AttributeTok{Data de admissão UTI}\StringTok{\textasciigrave{}}\NormalTok{))}
\end{Highlighting}
\end{Shaded}

\begin{verbatim}
##   skim_type skim_variable n_missing complete_rate POSIXct.min POSIXct.max
## 1   POSIXct          data         0             1  2020-03-30  2020-06-30
##   POSIXct.median POSIXct.n_unique
## 1     2020-05-21               86
\end{verbatim}

\begin{Shaded}
\begin{Highlighting}[]
\FunctionTok{mean}\NormalTok{(}\FunctionTok{difftime}\NormalTok{(df\_pre}\SpecialCharTok{$}\StringTok{\textasciigrave{}}\AttributeTok{Data de admissão UTI}\StringTok{\textasciigrave{}}\NormalTok{, df\_pre}\SpecialCharTok{$}\StringTok{\textasciigrave{}}\AttributeTok{Data de admissão hospitalar no HC}\StringTok{\textasciigrave{}}\NormalTok{, }\AttributeTok{units =} \StringTok{"days"}\NormalTok{))}
\end{Highlighting}
\end{Shaded}

\begin{verbatim}
## Time difference of 2.47 days
\end{verbatim}

\hypertarget{grupo-controle-2}{%
\paragraph{\texorpdfstring{\textbf{Grupo
controle}}{Grupo controle}}\label{grupo-controle-2}}

\begin{Shaded}
\begin{Highlighting}[]
\FunctionTok{as.data.frame}\NormalTok{(}\FunctionTok{skim}\NormalTok{(df\_npre}\SpecialCharTok{$}\StringTok{\textasciigrave{}}\AttributeTok{Data de admissão hospitalar no HC}\StringTok{\textasciigrave{}}\NormalTok{))}
\end{Highlighting}
\end{Shaded}

\begin{verbatim}
##   skim_type skim_variable n_missing complete_rate POSIXct.min POSIXct.max
## 1   POSIXct          data         0             1  2020-03-30  2020-06-30
##   POSIXct.median POSIXct.n_unique
## 1     2020-05-10               93
\end{verbatim}

\begin{Shaded}
\begin{Highlighting}[]
\FunctionTok{as.data.frame}\NormalTok{(}\FunctionTok{skim}\NormalTok{(df\_npre}\SpecialCharTok{$}\StringTok{\textasciigrave{}}\AttributeTok{Data de admissão UTI}\StringTok{\textasciigrave{}}\NormalTok{))}
\end{Highlighting}
\end{Shaded}

\begin{verbatim}
##   skim_type skim_variable n_missing complete_rate POSIXct.min POSIXct.max
## 1   POSIXct          data         0             1  2020-03-30  2020-06-30
##   POSIXct.median POSIXct.n_unique
## 1     2020-05-11               93
\end{verbatim}

\begin{Shaded}
\begin{Highlighting}[]
\FunctionTok{mean}\NormalTok{(}\FunctionTok{difftime}\NormalTok{(df\_npre}\SpecialCharTok{$}\StringTok{\textasciigrave{}}\AttributeTok{Data de admissão UTI}\StringTok{\textasciigrave{}}\NormalTok{, df\_npre}\SpecialCharTok{$}\StringTok{\textasciigrave{}}\AttributeTok{Data de admissão hospitalar no HC}\StringTok{\textasciigrave{}}\NormalTok{, }\AttributeTok{units =} \StringTok{"days"}\NormalTok{))}
\end{Highlighting}
\end{Shaded}

\begin{verbatim}
## Time difference of 0.988 days
\end{verbatim}

\hypertarget{anuxe1lise-saps}{%
\subsubsection{\texorpdfstring{\textbf{Análise
SAPS}}{Análise SAPS}}\label{anuxe1lise-saps}}

\hypertarget{grupo-cnafvni-3}{%
\paragraph{\texorpdfstring{\textbf{Grupo
CnAF/VNI}}{Grupo CnAF/VNI}}\label{grupo-cnafvni-3}}

\begin{Shaded}
\begin{Highlighting}[]
\FunctionTok{as.data.frame}\NormalTok{(}\FunctionTok{skim}\NormalTok{(df\_pre}\SpecialCharTok{$}\StringTok{\textasciigrave{}}\AttributeTok{SAPS 3 da admissão}\StringTok{\textasciigrave{}}\NormalTok{))}
\end{Highlighting}
\end{Shaded}

\begin{verbatim}
##   skim_type skim_variable n_missing complete_rate numeric.mean numeric.sd
## 1   numeric          data         0             1         57.3       14.6
##   numeric.p0 numeric.p25 numeric.p50 numeric.p75 numeric.p100
## 1         12          46          55          68          106
##                               numeric.hist
## 1 <U+2581><U+2586><U+2587><U+2585><U+2581>
\end{verbatim}

\begin{Shaded}
\begin{Highlighting}[]
\FunctionTok{ggplot}\NormalTok{(}\AttributeTok{data=}\NormalTok{df\_pre, }\FunctionTok{aes}\NormalTok{(}\AttributeTok{x=}\StringTok{\textasciigrave{}}\AttributeTok{SAPS 3 da admissão}\StringTok{\textasciigrave{}}\NormalTok{))}\SpecialCharTok{+}
  \FunctionTok{geom\_density}\NormalTok{(}\AttributeTok{fill=}\StringTok{"blue"}\NormalTok{)}
\end{Highlighting}
\end{Shaded}

\includegraphics{relatorio_epicov_pre_iot_files/figure-latex/unnamed-chunk-16-1.pdf}

\begin{Shaded}
\begin{Highlighting}[]
\FunctionTok{shapiro.test}\NormalTok{(df\_pre}\SpecialCharTok{$}\StringTok{\textasciigrave{}}\AttributeTok{SAPS 3 da admissão}\StringTok{\textasciigrave{}}\NormalTok{)}
\end{Highlighting}
\end{Shaded}

\begin{verbatim}
## 
##  Shapiro-Wilk normality test
## 
## data:  df_pre$`SAPS 3 da admissão`
## W = 1, p-value = 4e-05
\end{verbatim}

\hypertarget{grupo-controle-3}{%
\paragraph{\texorpdfstring{\textbf{Grupo
controle}}{Grupo controle}}\label{grupo-controle-3}}

\begin{Shaded}
\begin{Highlighting}[]
\FunctionTok{as.data.frame}\NormalTok{(}\FunctionTok{skim}\NormalTok{(df\_npre}\SpecialCharTok{$}\StringTok{\textasciigrave{}}\AttributeTok{SAPS 3 da admissão}\StringTok{\textasciigrave{}}\NormalTok{))}
\end{Highlighting}
\end{Shaded}

\begin{verbatim}
##   skim_type skim_variable n_missing complete_rate numeric.mean numeric.sd
## 1   numeric          data         1         0.999         67.1       16.6
##   numeric.p0 numeric.p25 numeric.p50 numeric.p75 numeric.p100
## 1         29          55          66          78          122
##                               numeric.hist
## 1 <U+2583><U+2587><U+2587><U+2583><U+2581>
\end{verbatim}

\begin{Shaded}
\begin{Highlighting}[]
\FunctionTok{ggplot}\NormalTok{(}\AttributeTok{data=}\NormalTok{df\_npre, }\FunctionTok{aes}\NormalTok{(}\AttributeTok{x=}\StringTok{\textasciigrave{}}\AttributeTok{SAPS 3 da admissão}\StringTok{\textasciigrave{}}\NormalTok{))}\SpecialCharTok{+}
  \FunctionTok{geom\_density}\NormalTok{(}\AttributeTok{fill=}\StringTok{"blue"}\NormalTok{)}
\end{Highlighting}
\end{Shaded}

\begin{verbatim}
## Warning: Removed 1 rows containing non-finite values (stat_density).
\end{verbatim}

\includegraphics{relatorio_epicov_pre_iot_files/figure-latex/unnamed-chunk-17-1.pdf}

\begin{Shaded}
\begin{Highlighting}[]
\FunctionTok{shapiro.test}\NormalTok{(df\_npre}\SpecialCharTok{$}\StringTok{\textasciigrave{}}\AttributeTok{SAPS 3 da admissão}\StringTok{\textasciigrave{}}\NormalTok{)}
\end{Highlighting}
\end{Shaded}

\begin{verbatim}
## 
##  Shapiro-Wilk normality test
## 
## data:  df_npre$`SAPS 3 da admissão`
## W = 1, p-value = 0.002
\end{verbatim}

\hypertarget{comparauxe7uxe3o-saps-3}{%
\paragraph{\texorpdfstring{\textbf{Comparação SAPS
3}}{Comparação SAPS 3}}\label{comparauxe7uxe3o-saps-3}}

\begin{Shaded}
\begin{Highlighting}[]
\FunctionTok{ggplot}\NormalTok{(}\AttributeTok{data=}\NormalTok{df\_pret, }\FunctionTok{aes}\NormalTok{(}\AttributeTok{x=}\StringTok{\textasciigrave{}}\AttributeTok{SAPS 3 da admissão}\StringTok{\textasciigrave{}}\NormalTok{, }\AttributeTok{fill=}\NormalTok{Grupo))}\SpecialCharTok{+}
  \FunctionTok{geom\_density}\NormalTok{(}\AttributeTok{alpha=}\FloatTok{0.2}\NormalTok{)}
\end{Highlighting}
\end{Shaded}

\begin{verbatim}
## Warning: Removed 1 rows containing non-finite values (stat_density).
\end{verbatim}

\includegraphics{relatorio_epicov_pre_iot_files/figure-latex/unnamed-chunk-18-1.pdf}

\begin{Shaded}
\begin{Highlighting}[]
\FunctionTok{wilcox.test}\NormalTok{(df\_pre}\SpecialCharTok{$}\StringTok{\textasciigrave{}}\AttributeTok{SAPS 3 da admissão}\StringTok{\textasciigrave{}}\NormalTok{, df\_npre}\SpecialCharTok{$}\StringTok{\textasciigrave{}}\AttributeTok{SAPS 3 da admissão}\StringTok{\textasciigrave{}}\NormalTok{)}
\end{Highlighting}
\end{Shaded}

\begin{verbatim}
## 
##  Wilcoxon rank sum test with continuity correction
## 
## data:  df_pre$`SAPS 3 da admissão` and df_npre$`SAPS 3 da admissão`
## W = 71109, p-value <2e-16
## alternative hypothesis: true location shift is not equal to 0
\end{verbatim}

\hypertarget{conclusuxe3o}{%
\paragraph{\texorpdfstring{\textbf{Conclusão:}}{Conclusão:}}\label{conclusuxe3o}}

\hypertarget{ambos-os-grupos-possuem-distrbuiuxe7uxe3o-nuxe3o-paramuxe9trica}{%
\paragraph{- Ambos os grupos possuem distrbuição não
paramétrica;}\label{ambos-os-grupos-possuem-distrbuiuxe7uxe3o-nuxe3o-paramuxe9trica}}

\hypertarget{o-grupo-intervenuxe7uxe3o-possui-distribuiuxe7uxe3o-bimodal}{%
\paragraph{- O grupo intervenção possui distribuição
bimodal;}\label{o-grupo-intervenuxe7uxe3o-possui-distribuiuxe7uxe3o-bimodal}}

\hypertarget{o-grupo-controle-possui-maior-saps-3-na-admissuxe3o-o-resultado-foi-estatuxedsticamente-significativo-portanto-podemos-concluir-que-o-grupo-controle-possuuxeda-maior-gravidade-na-admissuxe3o-em-uti.-talvez-porque-parte-desses-pacientes-juxe1-chegaram-uxe0-uti-intubados.}{%
\paragraph{- O grupo controle possui maior SAPS 3 na admissão, o
resultado foi estatísticamente significativo, portanto, podemos concluir
que o grupo controle possuía maior gravidade na admissão em UTI. Talvez
porque parte desses pacientes já chegaram à UTI
intubados.}\label{o-grupo-controle-possui-maior-saps-3-na-admissuxe3o-o-resultado-foi-estatuxedsticamente-significativo-portanto-podemos-concluir-que-o-grupo-controle-possuuxeda-maior-gravidade-na-admissuxe3o-em-uti.-talvez-porque-parte-desses-pacientes-juxe1-chegaram-uxe0-uti-intubados.}}

\hypertarget{saps-3-grupo-controle-que-nuxe3o-chegou-intubado-uxe0-uti}{%
\paragraph{\texorpdfstring{\textbf{SAPS 3 grupo controle que não chegou
intubado à
UTI}}{SAPS 3 grupo controle que não chegou intubado à UTI}}\label{saps-3-grupo-controle-que-nuxe3o-chegou-intubado-uxe0-uti}}

\begin{Shaded}
\begin{Highlighting}[]
\FunctionTok{as.data.frame}\NormalTok{(}\FunctionTok{skim}\NormalTok{(df\_nnpre}\SpecialCharTok{$}\StringTok{\textasciigrave{}}\AttributeTok{SAPS 3 da admissão}\StringTok{\textasciigrave{}}\NormalTok{))}
\end{Highlighting}
\end{Shaded}

\begin{verbatim}
##   skim_type skim_variable n_missing complete_rate numeric.mean numeric.sd
## 1   numeric          data         0             1         55.9       15.2
##   numeric.p0 numeric.p25 numeric.p50 numeric.p75 numeric.p100
## 1         29          45          55          64          109
##                               numeric.hist
## 1 <U+2586><U+2587><U+2585><U+2582><U+2581>
\end{verbatim}

\begin{Shaded}
\begin{Highlighting}[]
\FunctionTok{ggplot}\NormalTok{(}\AttributeTok{data=}\NormalTok{df\_nnpre, }\FunctionTok{aes}\NormalTok{(}\AttributeTok{x=}\StringTok{\textasciigrave{}}\AttributeTok{SAPS 3 da admissão}\StringTok{\textasciigrave{}}\NormalTok{))}\SpecialCharTok{+}
  \FunctionTok{geom\_density}\NormalTok{(}\AttributeTok{fill=}\StringTok{"blue"}\NormalTok{)}
\end{Highlighting}
\end{Shaded}

\includegraphics{relatorio_epicov_pre_iot_files/figure-latex/unnamed-chunk-19-1.pdf}

\begin{Shaded}
\begin{Highlighting}[]
\FunctionTok{shapiro.test}\NormalTok{(df\_nnpre}\SpecialCharTok{$}\StringTok{\textasciigrave{}}\AttributeTok{SAPS 3 da admissão}\StringTok{\textasciigrave{}}\NormalTok{)}
\end{Highlighting}
\end{Shaded}

\begin{verbatim}
## 
##  Shapiro-Wilk normality test
## 
## data:  df_nnpre$`SAPS 3 da admissão`
## W = 1, p-value = 1e-04
\end{verbatim}

\hypertarget{comparauxe7uxe3o-grupo-cnafvni-vs-grupo-controle-que-nuxe3o-chegou-intubado-uxe0-uti}{%
\paragraph{\texorpdfstring{\textbf{Comparação grupo CnAF/VNI vs grupo
controle que não chegou intubado à
UTI}}{Comparação grupo CnAF/VNI vs grupo controle que não chegou intubado à UTI}}\label{comparauxe7uxe3o-grupo-cnafvni-vs-grupo-controle-que-nuxe3o-chegou-intubado-uxe0-uti}}

\begin{Shaded}
\begin{Highlighting}[]
\FunctionTok{ggplot}\NormalTok{(}\AttributeTok{data=}\NormalTok{df\_prett, }\FunctionTok{aes}\NormalTok{(}\AttributeTok{x=}\StringTok{\textasciigrave{}}\AttributeTok{SAPS 3 da admissão}\StringTok{\textasciigrave{}}\NormalTok{, }\AttributeTok{fill=}\NormalTok{Grupo))}\SpecialCharTok{+}
  \FunctionTok{geom\_density}\NormalTok{(}\AttributeTok{alpha=}\FloatTok{0.2}\NormalTok{)}
\end{Highlighting}
\end{Shaded}

\includegraphics{relatorio_epicov_pre_iot_files/figure-latex/unnamed-chunk-20-1.pdf}

\begin{Shaded}
\begin{Highlighting}[]
\FunctionTok{wilcox.test}\NormalTok{(df\_pre}\SpecialCharTok{$}\StringTok{\textasciigrave{}}\AttributeTok{SAPS 3 da admissão}\StringTok{\textasciigrave{}}\NormalTok{, df\_nnpre}\SpecialCharTok{$}\StringTok{\textasciigrave{}}\AttributeTok{SAPS 3 da admissão}\StringTok{\textasciigrave{}}\NormalTok{)}
\end{Highlighting}
\end{Shaded}

\begin{verbatim}
## 
##  Wilcoxon rank sum test with continuity correction
## 
## data:  df_pre$`SAPS 3 da admissão` and df_nnpre$`SAPS 3 da admissão`
## W = 28323, p-value = 0.3
## alternative hypothesis: true location shift is not equal to 0
\end{verbatim}

\hypertarget{agora-podemos-concluir-que-quando-comparados-apenas-os-paciente-que-nuxe3o-chegaram-intubados-uxe0-uti-o-grupo-que-foi-tratado-com-cnafvni-possuia-gravidade-similar-uxe0-dos-pacientes-do-grupo-controle.}{%
\paragraph{Agora podemos concluir que quando comparados apenas os
paciente que não chegaram intubados à UTI, o grupo que foi tratado com
CnAF/VNI possuia gravidade similar à dos pacientes do grupo
controle.}\label{agora-podemos-concluir-que-quando-comparados-apenas-os-paciente-que-nuxe3o-chegaram-intubados-uxe0-uti-o-grupo-que-foi-tratado-com-cnafvni-possuia-gravidade-similar-uxe0-dos-pacientes-do-grupo-controle.}}

\hypertarget{necessidade-de-intubauxe7uxe3o}{%
\subsection{\texorpdfstring{\textbf{Necessidade de
intubação}}{Necessidade de intubação}}\label{necessidade-de-intubauxe7uxe3o}}

\hypertarget{grupo-cnafvni-4}{%
\paragraph{\texorpdfstring{\textbf{Grupo
CnAF/VNI}}{Grupo CnAF/VNI}}\label{grupo-cnafvni-4}}

\begin{Shaded}
\begin{Highlighting}[]
\FunctionTok{table}\NormalTok{(df\_pre}\SpecialCharTok{$}\StringTok{\textasciigrave{}}\AttributeTok{Teve falencia da VNI pre IOT?}\StringTok{\textasciigrave{}}\NormalTok{)}
\end{Highlighting}
\end{Shaded}

\begin{verbatim}
## 
##  No Yes 
##  81 151
\end{verbatim}

\begin{Shaded}
\begin{Highlighting}[]
\FunctionTok{table}\NormalTok{(df\_pre}\SpecialCharTok{$}\StringTok{\textasciigrave{}}\AttributeTok{Teve falencia do CNAF pre IOT?}\StringTok{\textasciigrave{}}\NormalTok{)}
\end{Highlighting}
\end{Shaded}

\begin{verbatim}
## 
##  No Yes 
##  41  58
\end{verbatim}

\begin{Shaded}
\begin{Highlighting}[]
\NormalTok{t1 }\OtherTok{\textless{}{-}} \FunctionTok{select}\NormalTok{(df\_pre, }\StringTok{\textasciigrave{}}\AttributeTok{IRpA que precisou de VM}\StringTok{\textasciigrave{}}\NormalTok{, Grupo)}
\FunctionTok{ggplot}\NormalTok{(}\AttributeTok{data=}\NormalTok{df\_pre, }\FunctionTok{aes}\NormalTok{(}\AttributeTok{x=}\StringTok{\textasciigrave{}}\AttributeTok{IRpA que precisou de VM}\StringTok{\textasciigrave{}}\NormalTok{, }\AttributeTok{y=}\NormalTok{(..count..)}\SpecialCharTok{*}\DecValTok{100}\SpecialCharTok{/}\FunctionTok{sum}\NormalTok{(..count..), }\AttributeTok{fill=}\StringTok{\textasciigrave{}}\AttributeTok{IRpA que precisou de VM}\StringTok{\textasciigrave{}}\NormalTok{)) }\SpecialCharTok{+}
  \FunctionTok{geom\_bar}\NormalTok{() }\SpecialCharTok{+}
  \FunctionTok{ylab}\NormalTok{(}\StringTok{"Porcentagem de pacientes (\%)"}\NormalTok{) }\SpecialCharTok{+}
  \FunctionTok{xlab}\NormalTok{(}\StringTok{"Necessidade de intubação (Grupo CnAF/VNI)"}\NormalTok{)}
\end{Highlighting}
\end{Shaded}

\includegraphics{relatorio_epicov_pre_iot_files/figure-latex/unnamed-chunk-21-1.pdf}

\hypertarget{grupo-controle-4}{%
\paragraph{\texorpdfstring{\textbf{Grupo
Controle}}{Grupo Controle}}\label{grupo-controle-4}}

\begin{Shaded}
\begin{Highlighting}[]
\NormalTok{t2 }\OtherTok{\textless{}{-}} \FunctionTok{select}\NormalTok{(df\_nnpre, }\StringTok{\textasciigrave{}}\AttributeTok{IRpA que precisou de VM}\StringTok{\textasciigrave{}}\NormalTok{, Grupo)}
\FunctionTok{ggplot}\NormalTok{(}\AttributeTok{data=}\NormalTok{df\_nnpre, }\FunctionTok{aes}\NormalTok{(}\AttributeTok{x=}\StringTok{\textasciigrave{}}\AttributeTok{IRpA que precisou de VM}\StringTok{\textasciigrave{}}\NormalTok{, }\AttributeTok{y=}\NormalTok{(..count..)}\SpecialCharTok{*}\DecValTok{100}\SpecialCharTok{/}\FunctionTok{sum}\NormalTok{(..count..), }\AttributeTok{fill=}\StringTok{\textasciigrave{}}\AttributeTok{IRpA que precisou de VM}\StringTok{\textasciigrave{}}\NormalTok{)) }\SpecialCharTok{+}
  \FunctionTok{geom\_bar}\NormalTok{()}\SpecialCharTok{+}
  \FunctionTok{ylab}\NormalTok{(}\StringTok{"Porcentagem de pacientes (\%)"}\NormalTok{) }\SpecialCharTok{+}
  \FunctionTok{xlab}\NormalTok{(}\StringTok{"Necessidade de intubação (Grupo controle"}\NormalTok{)}
\end{Highlighting}
\end{Shaded}

\includegraphics{relatorio_epicov_pre_iot_files/figure-latex/unnamed-chunk-22-1.pdf}

\hypertarget{comparauxe7uxe3o---teste-chi-quadrado}{%
\paragraph{\texorpdfstring{\textbf{Comparação - Teste
chi-quadrado}}{Comparação - Teste chi-quadrado}}\label{comparauxe7uxe3o---teste-chi-quadrado}}

\begin{Shaded}
\begin{Highlighting}[]
\NormalTok{t3 }\OtherTok{\textless{}{-}} \FunctionTok{rbind}\NormalTok{(t1, t2)}
\FunctionTok{chisq.test}\NormalTok{(t3}\SpecialCharTok{$}\NormalTok{Grupo, t3}\SpecialCharTok{$}\StringTok{\textasciigrave{}}\AttributeTok{IRpA que precisou de VM}\StringTok{\textasciigrave{}}\NormalTok{)}
\end{Highlighting}
\end{Shaded}

\begin{verbatim}
## 
##  Pearson's Chi-squared test with Yates' continuity correction
## 
## data:  t3$Grupo and t3$`IRpA que precisou de VM`
## X-squared = 51, df = 1, p-value = 1e-12
\end{verbatim}

\hypertarget{desfecho-uti}{%
\subsubsection{\texorpdfstring{\textbf{Desfecho
UTI}}{Desfecho UTI}}\label{desfecho-uti}}

\hypertarget{grupo-cnafvni-5}{%
\paragraph{\texorpdfstring{\textbf{Grupo
CnAF/VNI}}{Grupo CnAF/VNI}}\label{grupo-cnafvni-5}}

\begin{Shaded}
\begin{Highlighting}[]
\FunctionTok{as.data.frame}\NormalTok{(}\FunctionTok{table}\NormalTok{(df\_pre}\SpecialCharTok{$}\StringTok{\textasciigrave{}}\AttributeTok{Desfecho na UTI}\StringTok{\textasciigrave{}}\NormalTok{))}
\end{Highlighting}
\end{Shaded}

\begin{verbatim}
##                             Var1 Freq
## 1                 Alta para casa    4
## 2           Alta para enfermaria  136
## 3                          Óbito  115
## 4 Permanece na UTI em 28/07/2020    4
## 5           Transferência de UTI   12
\end{verbatim}

\begin{Shaded}
\begin{Highlighting}[]
\FunctionTok{as.data.frame}\NormalTok{(}\FunctionTok{prop.table}\NormalTok{(}\FunctionTok{table}\NormalTok{(df\_pre}\SpecialCharTok{$}\StringTok{\textasciigrave{}}\AttributeTok{Desfecho na UTI}\StringTok{\textasciigrave{}}\NormalTok{)))}
\end{Highlighting}
\end{Shaded}

\begin{verbatim}
##                             Var1   Freq
## 1                 Alta para casa 0.0148
## 2           Alta para enfermaria 0.5018
## 3                          Óbito 0.4244
## 4 Permanece na UTI em 28/07/2020 0.0148
## 5           Transferência de UTI 0.0443
\end{verbatim}

\begin{Shaded}
\begin{Highlighting}[]
\FunctionTok{ggplot}\NormalTok{(}\AttributeTok{data=}\NormalTok{df\_pre, }\FunctionTok{aes}\NormalTok{(}\AttributeTok{x=}\StringTok{\textasciigrave{}}\AttributeTok{Desfecho na UTI}\StringTok{\textasciigrave{}}\NormalTok{, }\AttributeTok{y=}\NormalTok{(..count..)}\SpecialCharTok{*}\DecValTok{100}\SpecialCharTok{/}\FunctionTok{sum}\NormalTok{(..count..), }\AttributeTok{fill=}\StringTok{\textasciigrave{}}\AttributeTok{Desfecho na UTI}\StringTok{\textasciigrave{}}\NormalTok{)) }\SpecialCharTok{+}
  \FunctionTok{geom\_bar}\NormalTok{() }\SpecialCharTok{+}
  \FunctionTok{ylab}\NormalTok{(}\StringTok{"Porcentagem de pacientes (\%)"}\NormalTok{) }\SpecialCharTok{+}
  \FunctionTok{xlab}\NormalTok{(}\StringTok{"Desfecho UTI (CnAF/VNI)"}\NormalTok{) }\SpecialCharTok{+}
  \FunctionTok{theme}\NormalTok{(}\AttributeTok{axis.text.x =} \FunctionTok{element\_text}\NormalTok{(}\AttributeTok{angle =} \DecValTok{90}\NormalTok{, }\AttributeTok{vjust =} \FloatTok{0.5}\NormalTok{, }\AttributeTok{hjust=}\DecValTok{1}\NormalTok{))}
\end{Highlighting}
\end{Shaded}

\includegraphics{relatorio_epicov_pre_iot_files/figure-latex/unnamed-chunk-24-1.pdf}

\hypertarget{grupo-controle-5}{%
\paragraph{\texorpdfstring{\textbf{Grupo
controle}}{Grupo controle}}\label{grupo-controle-5}}

\begin{Shaded}
\begin{Highlighting}[]
\FunctionTok{as.data.frame}\NormalTok{(}\FunctionTok{table}\NormalTok{(df\_nnpre}\SpecialCharTok{$}\StringTok{\textasciigrave{}}\AttributeTok{Desfecho na UTI}\StringTok{\textasciigrave{}}\NormalTok{))}
\end{Highlighting}
\end{Shaded}

\begin{verbatim}
##                   Var1 Freq
## 1       Alta para casa    2
## 2 Alta para enfermaria  140
## 3                Óbito   45
## 4 Transferência de UTI   10
\end{verbatim}

\begin{Shaded}
\begin{Highlighting}[]
\FunctionTok{as.data.frame}\NormalTok{(}\FunctionTok{prop.table}\NormalTok{(}\FunctionTok{table}\NormalTok{(df\_nnpre}\SpecialCharTok{$}\StringTok{\textasciigrave{}}\AttributeTok{Desfecho na UTI}\StringTok{\textasciigrave{}}\NormalTok{)))}
\end{Highlighting}
\end{Shaded}

\begin{verbatim}
##                   Var1   Freq
## 1       Alta para casa 0.0102
## 2 Alta para enfermaria 0.7107
## 3                Óbito 0.2284
## 4 Transferência de UTI 0.0508
\end{verbatim}

\begin{Shaded}
\begin{Highlighting}[]
\FunctionTok{ggplot}\NormalTok{(}\AttributeTok{data=}\NormalTok{df\_nnpre, }\FunctionTok{aes}\NormalTok{(}\AttributeTok{x=}\StringTok{\textasciigrave{}}\AttributeTok{Desfecho na UTI}\StringTok{\textasciigrave{}}\NormalTok{, }\AttributeTok{y=}\NormalTok{(..count..)}\SpecialCharTok{*}\DecValTok{100}\SpecialCharTok{/}\FunctionTok{sum}\NormalTok{(..count..), }\AttributeTok{fill=}\StringTok{\textasciigrave{}}\AttributeTok{Desfecho na UTI}\StringTok{\textasciigrave{}}\NormalTok{)) }\SpecialCharTok{+}
  \FunctionTok{geom\_bar}\NormalTok{()}\SpecialCharTok{+}
  \FunctionTok{ylab}\NormalTok{(}\StringTok{"Porcentagem de pacientes (\%)"}\NormalTok{) }\SpecialCharTok{+}
  \FunctionTok{xlab}\NormalTok{(}\StringTok{"Desfecho UTI (Controle)"}\NormalTok{) }\SpecialCharTok{+}
  \FunctionTok{theme}\NormalTok{(}\AttributeTok{axis.text.x =} \FunctionTok{element\_text}\NormalTok{(}\AttributeTok{angle =} \DecValTok{90}\NormalTok{, }\AttributeTok{vjust =} \FloatTok{0.5}\NormalTok{, }\AttributeTok{hjust=}\DecValTok{1}\NormalTok{))}
\end{Highlighting}
\end{Shaded}

\includegraphics{relatorio_epicov_pre_iot_files/figure-latex/unnamed-chunk-25-1.pdf}

\hypertarget{comparauxe7uxe3o---teste-chi-quadrado-1}{%
\paragraph{\texorpdfstring{\textbf{Comparação - Teste
chi-quadrado}}{Comparação - Teste chi-quadrado}}\label{comparauxe7uxe3o---teste-chi-quadrado-1}}

\begin{Shaded}
\begin{Highlighting}[]
\NormalTok{t1 }\OtherTok{\textless{}{-}} \FunctionTok{select}\NormalTok{(df\_pre, }\StringTok{\textasciigrave{}}\AttributeTok{Desfecho na UTI}\StringTok{\textasciigrave{}}\NormalTok{, Grupo)}
  
\NormalTok{t2 }\OtherTok{\textless{}{-}} \FunctionTok{select}\NormalTok{(df\_nnpre, }\StringTok{\textasciigrave{}}\AttributeTok{Desfecho na UTI}\StringTok{\textasciigrave{}}\NormalTok{, Grupo)}

\NormalTok{t3 }\OtherTok{\textless{}{-}} \FunctionTok{rbind}\NormalTok{(t1, t2)}
\FunctionTok{chisq.test}\NormalTok{(t3}\SpecialCharTok{$}\NormalTok{Grupo, t3}\SpecialCharTok{$}\StringTok{\textasciigrave{}}\AttributeTok{Desfecho na UTI}\StringTok{\textasciigrave{}}\NormalTok{)}
\end{Highlighting}
\end{Shaded}

\begin{verbatim}
## Warning in chisq.test(t3$Grupo, t3$`Desfecho na UTI`): Chi-squared approximation
## may be incorrect
\end{verbatim}

\begin{verbatim}
## 
##  Pearson's Chi-squared test
## 
## data:  t3$Grupo and t3$`Desfecho na UTI`
## X-squared = 24, df = 4, p-value = 7e-05
\end{verbatim}

\hypertarget{consequente-uxe0-maior-necessidade-de-intubauxe7uxe3o-observamos-maior-nuxfamero-de-uxf3bitos-no-grupo-cnafvni.}{%
\paragraph{Consequente à maior necessidade de intubação, observamos
maior número de óbitos no grupo
CnAF/VNI.}\label{consequente-uxe0-maior-necessidade-de-intubauxe7uxe3o-observamos-maior-nuxfamero-de-uxf3bitos-no-grupo-cnafvni.}}

\hypertarget{desfecho-hospitalar}{%
\subsubsection{\texorpdfstring{\textbf{Desfecho
hospitalar}}{Desfecho hospitalar}}\label{desfecho-hospitalar}}

\hypertarget{grupo-cnafvni-6}{%
\paragraph{\texorpdfstring{\textbf{Grupo
CnAF/VNI}}{Grupo CnAF/VNI}}\label{grupo-cnafvni-6}}

\begin{Shaded}
\begin{Highlighting}[]
\FunctionTok{as.data.frame}\NormalTok{(}\FunctionTok{table}\NormalTok{(df\_pre}\SpecialCharTok{$}\StringTok{\textasciigrave{}}\AttributeTok{Desfecho Hospitalar}\StringTok{\textasciigrave{}}\NormalTok{))}
\end{Highlighting}
\end{Shaded}

\begin{verbatim}
##                                  Var1 Freq
## 1                     Alta hospitalar  109
## 2                               Óbito  127
## 3 Permanece no hospital em 28/07/2020   15
## 4          Transferência de  Hospital   20
\end{verbatim}

\begin{Shaded}
\begin{Highlighting}[]
\FunctionTok{as.data.frame}\NormalTok{(}\FunctionTok{prop.table}\NormalTok{(}\FunctionTok{table}\NormalTok{(df\_pre}\SpecialCharTok{$}\StringTok{\textasciigrave{}}\AttributeTok{Desfecho Hospitalar}\StringTok{\textasciigrave{}}\NormalTok{)))}
\end{Highlighting}
\end{Shaded}

\begin{verbatim}
##                                  Var1   Freq
## 1                     Alta hospitalar 0.4022
## 2                               Óbito 0.4686
## 3 Permanece no hospital em 28/07/2020 0.0554
## 4          Transferência de  Hospital 0.0738
\end{verbatim}

\begin{Shaded}
\begin{Highlighting}[]
\FunctionTok{ggplot}\NormalTok{(}\AttributeTok{data=}\NormalTok{df\_pre, }\FunctionTok{aes}\NormalTok{(}\AttributeTok{x=}\StringTok{\textasciigrave{}}\AttributeTok{Desfecho Hospitalar}\StringTok{\textasciigrave{}}\NormalTok{, }\AttributeTok{y=}\NormalTok{(..count..)}\SpecialCharTok{*}\DecValTok{100}\SpecialCharTok{/}\FunctionTok{sum}\NormalTok{(..count..), }\AttributeTok{fill=}\StringTok{\textasciigrave{}}\AttributeTok{Desfecho Hospitalar}\StringTok{\textasciigrave{}}\NormalTok{)) }\SpecialCharTok{+}
  \FunctionTok{geom\_bar}\NormalTok{() }\SpecialCharTok{+}
  \FunctionTok{ylab}\NormalTok{(}\StringTok{"Porcentagem de pacientes (\%)"}\NormalTok{) }\SpecialCharTok{+}
  \FunctionTok{xlab}\NormalTok{(}\StringTok{"Desfecho Hospitalar (CnAF/VNI)"}\NormalTok{) }\SpecialCharTok{+}
  \FunctionTok{theme}\NormalTok{(}\AttributeTok{axis.text.x =} \FunctionTok{element\_text}\NormalTok{(}\AttributeTok{angle =} \DecValTok{90}\NormalTok{, }\AttributeTok{vjust =} \FloatTok{0.5}\NormalTok{, }\AttributeTok{hjust=}\DecValTok{1}\NormalTok{))}
\end{Highlighting}
\end{Shaded}

\includegraphics{relatorio_epicov_pre_iot_files/figure-latex/unnamed-chunk-27-1.pdf}

\hypertarget{grupo-controle-6}{%
\paragraph{\texorpdfstring{\textbf{Grupo
controle}}{Grupo controle}}\label{grupo-controle-6}}

\begin{Shaded}
\begin{Highlighting}[]
\FunctionTok{as.data.frame}\NormalTok{(}\FunctionTok{table}\NormalTok{(df\_nnpre}\SpecialCharTok{$}\StringTok{\textasciigrave{}}\AttributeTok{Desfecho Hospitalar}\StringTok{\textasciigrave{}}\NormalTok{))}
\end{Highlighting}
\end{Shaded}

\begin{verbatim}
##                                  Var1 Freq
## 1                     Alta hospitalar  114
## 2                               Óbito   57
## 3 Permanece no hospital em 28/07/2020    3
## 4          Transferência de  Hospital   23
\end{verbatim}

\begin{Shaded}
\begin{Highlighting}[]
\FunctionTok{as.data.frame}\NormalTok{(}\FunctionTok{prop.table}\NormalTok{(}\FunctionTok{table}\NormalTok{(df\_nnpre}\SpecialCharTok{$}\StringTok{\textasciigrave{}}\AttributeTok{Desfecho Hospitalar}\StringTok{\textasciigrave{}}\NormalTok{)))}
\end{Highlighting}
\end{Shaded}

\begin{verbatim}
##                                  Var1   Freq
## 1                     Alta hospitalar 0.5787
## 2                               Óbito 0.2893
## 3 Permanece no hospital em 28/07/2020 0.0152
## 4          Transferência de  Hospital 0.1168
\end{verbatim}

\begin{Shaded}
\begin{Highlighting}[]
\FunctionTok{ggplot}\NormalTok{(}\AttributeTok{data=}\NormalTok{df\_nnpre, }\FunctionTok{aes}\NormalTok{(}\AttributeTok{x=}\StringTok{\textasciigrave{}}\AttributeTok{Desfecho Hospitalar}\StringTok{\textasciigrave{}}\NormalTok{, }\AttributeTok{y=}\NormalTok{(..count..)}\SpecialCharTok{*}\DecValTok{100}\SpecialCharTok{/}\FunctionTok{sum}\NormalTok{(..count..), }\AttributeTok{fill=}\StringTok{\textasciigrave{}}\AttributeTok{Desfecho Hospitalar}\StringTok{\textasciigrave{}}\NormalTok{)) }\SpecialCharTok{+}
  \FunctionTok{geom\_bar}\NormalTok{()}\SpecialCharTok{+}
  \FunctionTok{ylab}\NormalTok{(}\StringTok{"Porcentagem de pacientes (\%)"}\NormalTok{) }\SpecialCharTok{+}
  \FunctionTok{xlab}\NormalTok{(}\StringTok{"Desfecho Hospitalar (Controle)"}\NormalTok{) }\SpecialCharTok{+}
  \FunctionTok{theme}\NormalTok{(}\AttributeTok{axis.text.x =} \FunctionTok{element\_text}\NormalTok{(}\AttributeTok{angle =} \DecValTok{90}\NormalTok{, }\AttributeTok{vjust =} \FloatTok{0.5}\NormalTok{, }\AttributeTok{hjust=}\DecValTok{1}\NormalTok{))}
\end{Highlighting}
\end{Shaded}

\includegraphics{relatorio_epicov_pre_iot_files/figure-latex/unnamed-chunk-28-1.pdf}

\hypertarget{comparauxe7uxe3o-1}{%
\paragraph{\texorpdfstring{\textbf{Comparação}}{Comparação}}\label{comparauxe7uxe3o-1}}

\begin{Shaded}
\begin{Highlighting}[]
\NormalTok{t1 }\OtherTok{\textless{}{-}} \FunctionTok{select}\NormalTok{(df\_pre, }\StringTok{\textasciigrave{}}\AttributeTok{Desfecho Hospitalar}\StringTok{\textasciigrave{}}\NormalTok{, Grupo)}
  
\NormalTok{t2 }\OtherTok{\textless{}{-}} \FunctionTok{select}\NormalTok{(df\_nnpre, }\StringTok{\textasciigrave{}}\AttributeTok{Desfecho Hospitalar}\StringTok{\textasciigrave{}}\NormalTok{, Grupo)}

\NormalTok{t3 }\OtherTok{\textless{}{-}} \FunctionTok{rbind}\NormalTok{(t1, t2)}
\FunctionTok{chisq.test}\NormalTok{(t3}\SpecialCharTok{$}\NormalTok{Grupo, t3}\SpecialCharTok{$}\StringTok{\textasciigrave{}}\AttributeTok{Desfecho Hospitalar}\StringTok{\textasciigrave{}}\NormalTok{)}
\end{Highlighting}
\end{Shaded}

\begin{verbatim}
## 
##  Pearson's Chi-squared test
## 
## data:  t3$Grupo and t3$`Desfecho Hospitalar`
## X-squared = 24, df = 3, p-value = 3e-05
\end{verbatim}

\end{document}
